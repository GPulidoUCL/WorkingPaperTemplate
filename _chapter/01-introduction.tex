\textcite{goldinGrandGenderConvergence2014a} highlights the need to look within occupations
to understand how jobs are organized and compensated and how this might diferentially affect men and
women. 

A commonly used measure to summarize differences in the distribution of women and men across occupation categories 
is the index of segregation developed by Duncan (1955). 

The index of occupational segregation by sex is computed as

\begin{equation}
    D = 0.5 \sum_j |M_j - F_j|,
\end{equation}

where $M_j$ ($F_j$) is the fraction of all employed males (females) who work in occupation $j$. 
The index, which ranges between zero and one, indicates the proportion of women or men that would need 
to change occupations for the occupational distribution of men and women to be the same. In other words, 
if the distribution of men and women across occupational categories were identical (complete integration), 
the segregation index would equal zero. If all the occupations were either completely male or completely female 
(complete segregation), the segregation index would equal one.


\begin{table}[!t]
    \centering
    \caption{Task Measures from the Skills and Employment Survey}
    \label{tab:label}
    % \resizebox*{\textwidth}{!}{
    \begin{threeparttable}
        % \setlength{\tabcolsep}{6pt}
        % \small
        \begin{tabular}{@{}p{4cm}p{11cm}@{}}
    \toprule
    \textbf{Skill} & \textbf{Task} \\
    \midrule
    
    \multirow{6}{*}{Literacy} 
    & Reading written information, e.g.\ forms, notices, or signs \\
    & Reading short documents, e.g.\ letters or memos \\
    & Reading long documents, e.g.\ long reports, manuals, etc. \\
    & Writing material such as forms, notices, or signs \\
    & Writing short documents, e.g.\ letters or memos \\
    & Writing long documents with correct spelling/grammar \\
    
    \midrule
    \multirow{3}{*}{Numeracy} 
    & Adding, subtracting, multiplying, or dividing numbers \\
    & Calculations using decimals, percentages, or fractions \\
    & More advanced mathematical or statistical procedures \\
    
    \midrule
    \multirow{3}{*}{Numeracy} 
    & Adding, subtracting, multiplying, or dividing numbers \\
    & Calculations using decimals, percentages, or fractions \\
    & More advanced mathematical or statistical procedures \\

    \midrule
    \multirow{4}{*}{Physical} 
    & Carrying, pushing or pulling heavy objects \\
    & Working for long periods on physical activities \\
    & mend, repair, assemble, construct or adjust things \\
    & knowledge of how to use or operate tools, equipment or machinery \\

    \midrule
    \multirow{5}{*}{\begin{tabular}[c]{@{}l@{}}Professional\\ communication\end{tabular}} 
    & Instructing, training, or teaching people \\
    & Persuading or influencing others \\
    & Making speeches or presentations \\
    & Planning the activities of others \\
    & Listening carefully to colleagues \\
    
    \midrule
    \multirow{4}{*}{Problem solving} 
    & Spotting problems or faults \\
    & Working out the cause of problems or faults \\
    & Thinking of solutions to problems \\
    & Analysing complex problems in depth \\
    
    \midrule
    \multirow{6}{*}{\begin{tabular}[c]{@{}l@{}}Computer use\\ complexity\end{tabular}} 
    & Importance of computer use and complexity of computer use: \\
    & Not at all = 0 \\
    & Straightforward use = 1 \\
    & Moderate use = 2 \\
    & Complex use = 3 \\
    & Advanced use = 4 \\
    
    \bottomrule
    \end{tabular}



    
        \begin{tablenotes}[flushleft]
            \scriptsize{\item \textit{Notes}: Notes here}
        \end{tablenotes}
    \end{threeparttable}
    % }
\end{table}
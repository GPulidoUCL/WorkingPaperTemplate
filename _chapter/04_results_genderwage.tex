
The empirical analysis estimates the gender wage gap using the following log-linear regression model:

\begin{equation}
\log(wage_i) = \alpha + \beta \cdot Female_i + \mathbf{X}_i'\boldsymbol{\gamma} + \delta_{j(i)} + \theta_{k(i)} + \varepsilon_i
\end{equation}

where $\log(wage_i)$ denotes the natural logarithm of hourly wages for individual $i$, and $Female_i$ is a binary 
indicator equal to one if the individual is female and zero otherwise. The vector $\mathbf{X}_i$ includes individual-level 
controls such as age, education, and region. The terms $\delta_{j(i)}$ and $\theta_{k(i)}$ represent fixed effects for occupation 
(3-digit SOC) and industry (2-digit SIC), respectively, and $\varepsilon_i$ is the error term. In extended specifications, the model 
incorporates additional variables capturing task and skill requirements derived from job-level data. The coefficient $\beta$ measures 
the conditional gender wage gap, interpreted as the average percentage difference in hourly wages between women and men, conditional 
on observed characteristics, occupational sorting, and task content.


Table X reports estimates of the gender wage gap from a series of linear regressions where the 
dependent variable is the log of hourly wages. Column (1) presents the baseline specification, 
which includes controls for individual-level characteristics such as age, education, and region. 
In this model, women earn approximately 14.9\% less than men. Column (2) adds controls for occupation and industry, reducing the estimated wage 
gap to 6.1\% and explaining 59.2\% of the initial difference. This substantial reduction highlights the 
central role of occupational and industrial sorting in shaping gender disparities in pay.

Columns (3) through (14) sequentially introduce a wide range of skill and task measures, including 
indicators of cognitive, manual, interpersonal, and planning-related demands. Across these models, 
the adjusted gender wage gap remains consistently between 5.4\% and 7.4\%, with relatively minor variation 
in the percent of the gap explained. The highest explanatory power is reached in column (12), where 63.6\% of 
the original gap is accounted for, while the lowest occurs in column (13), at 50.6\%. These changes, though 
measurable, are modest compared to the explanatory contribution of occupation and industry introduced in column (2).

Overall, the results suggest that while gender differences in task and skill requirements explain a small 
additional portion of the wage gap, the vast majority of the explained component derives from gender sorting 
into different occupations and industries. Even after accounting for detailed task characteristics, a 
persistent and statistically significant wage penalty of around 6\% remains for women, pointing to the 
enduring role of within-job and potentially discriminatory mechanisms in the gender wage structure.

\begin{table}[!t]
    \centering
    \caption{Gender Wage Gap}
    \label{tab:task-wagegap}
     \resizebox*{\textwidth}{!}{
    \begin{threeparttable}
        % \setlength{\tabcolsep}{6pt}
        % \small
        \begin{tabular}{@{}p{3.8cm}cccccccccccccc@{}}
    \toprule
     & (1) & (2) & (3) & (4) & (5) & (6) & (7) & (8) & (9) & (10) & (11) & (12) & (13) & (14) \\
    \midrule
    Female & -0.149*** & -0.061* & -0.063* & -0.060* & -0.059* & -0.064* & -0.056* & -0.060* & -0.061* & -0.061* & -0.064* & -0.054+ & -0.074** & -0.064* \\
           & (0.026)   & (0.028) & (0.028) & (0.028) & (0.028) & (0.028) & (0.027) & (0.028) & (0.028) & (0.028) & (0.027) & (0.029) & (0.027) & (0.028) \\
    Individual controls & Yes & Yes & Yes & Yes & Yes & Yes & Yes & Yes & Yes & Yes & Yes & Yes & Yes & Yes \\
    Occupation and Industry     & No  & Yes & Yes & Yes & Yes & Yes & Yes & Yes & Yes & Yes & Yes & Yes & Yes & Yes \\
    \% Gender Gap Explained                 & 0\% & 59.2\% & 57.6\% & 59.5\% & 60.4\% & 57.1\% & 62.4\% & 59.9\% & 59.1\% & 59.2\% & 57.3\% & 63.6\% & 50.6\% & 56.8\% \\
    Observations                            & 16,735,292 & 16,735,292 & 16,735,292 & 16,735,292 & 16,735,292 & 16,735,292 & 16,735,292 & 16,735,292 & 16,735,292 & 16,735,292 & 16,735,292 & 15,821,040 & 16,735,292 & 15,821,040 \\
    \bottomrule
    \end{tabular}
    
        \begin{tablenotes}[flushleft]
            \scriptsize{\item \textit{Source}: Adapted from \textcite{lindleyGenderDifferencesJob2015a}. 
            \item \textit{Notes}: Stadardt errors are clustered at the occupation level. + p < 0.1, * p < 0.05, ** p < 0.01, *** p < 0.001.}
        \end{tablenotes}
    \end{threeparttable}
     }
\end{table}
